\begin{table}
\centering
\setlength{\tabcolsep}{0.0pc}
\begin{tabular*}{\textwidth}{@{\extracolsep{\fill}}lccccc}
\noalign{\smallskip}\hline\noalign{\smallskip}
\textbf{Signal channel} &
$\langle\epsilon\mathrm{\sigma}\rangle_\mathrm{obs}^{95}$[fb] &
$S_\mathrm{obs}^{95}$
& $S_\mathrm{exp}^{95}$
& $\mathrm{CL_b}$ &
$p(s=0)$ ($Z$) \\
\noalign{\smallskip}\hline\noalign{\smallskip}
SS & $1.92$ & $9.0$ & $6.3^{+3.2}_{-1.8}$ & $0.78$ & $0.22~(0.79)$ \\
\noalign{\smallskip}\hline\noalign{\smallskip}
\end{tabular*}
\caption{
Model-independent fit results.
Left to right: the observed 95\% upper limit on the visible cross-section
$\langle\epsilon\sigma\rangle_\mathrm{obs}^{95}$,
its corresponding signal expectation $S_\mathrm{obs}^{95}$,
expected 95\% upper limits on the signal expectation $S_\mathrm{exp}^{95}$ as would be obtained
were the data the background expectation or its $\pm 1\sigma$ variations,
$\mathrm{CL_b}$ evaluated with the signal expectation set to its observed upper limit,
and the discovery $p$-value $p(s = 0)$ capped at 0.5,
with its equivalent significance.
Limits use the $\mathrm{CL_s}$ prescription.
Upper limits use the one-sided profile likelihood test statistic.
The discovery p-value uses a profile likelihood test statistic in a one-sided test.
All p-values are estimated by
sampling the test statistic distribution with nuisance parameters at their best-fit values.
}
\label{tab:results.discoxsec.SS}
\end{table}
